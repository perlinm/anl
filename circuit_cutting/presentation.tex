\documentclass[aspectratio=43,usenames,dvipsnames]{beamer}


%%% beamer theme
\usepackage{beamerthemelined}
\usetheme{default} % theme with outline sidebar on the right
\setbeamertemplate{headline}{} % clear header
\setbeamertemplate{navigation symbols}{} % clear navigation symbols
\setbeamertemplate{footline} % place frame number in footer
{
  \hbox{\begin{beamercolorbox}
      [wd=1\paperwidth,ht=2.25ex,dp=1ex,right]{framenumber}
      \usebeamercolor[fg]{subtitle}
      \insertframenumber{} / \inserttotalframenumber
      \hspace*{2ex}
    \end{beamercolorbox}}
}

%%% symbols, notations, etc.
\usepackage{physics,braket,bm,amssymb} % physics and math
\usepackage{mathtools} % for the \prescript command
\usefonttheme[onlymath]{serif} % "regular" math font

\renewcommand{\t}{\text} % text in math mode
\newcommand{\f}[2]{\dfrac{#1}{#2}} % shorthand for fractions
\newcommand{\p}[1]{\left(#1\right)} % parenthesis
\renewcommand{\sp}[1]{\left[#1\right]} % square parenthesis
\renewcommand{\set}[1]{\left\{#1\right\}} % curly parenthesis
\newcommand{\bk}{\Braket} % shorthand for braket notation
\renewcommand{\v}{\bm} % bold vectors
\newcommand{\uv}[1]{\bm{\hat{#1}}} % unit vectors
\renewcommand{\c}{\cdot} % inner product

\newcommand{\B}{\mathcal{B}}
\renewcommand{\P}{\mathcal{P}}

\renewcommand*{\thefootnote}{\alph{footnote}}

\usepackage{graphicx,float} % for including images and placing them
\graphicspath{{./figures/}} % set path for all figures


%%%%%%%%%%%%%%%%%%%%%%%%%%%%%%%%%%%%%%%%%%%%%%%%%%%%%%%%%%%%%%%%%%%%%%
%%% tensor network drawing tools
%%% taken in part from arxiv.org/abs/1603.03039

\usepackage{xcolor}
\definecolor{tensorblue}{rgb}{0.8,0.8,1}
\definecolor{tensorred}{rgb}{1,0.5,0.5}
\definecolor{tensorpurp}{rgb}{1,0.5,1}
\definecolor{tensorgreen}{rgb}{0.2,0.7,0.2}

\usepackage{tikz}
\newcommand{\diagram}[1]{
  \begin{tikzpicture}
    [scale=.5, every node/.style = {sloped,allow upside down},
    baseline = {([yshift=+0ex]current bounding box.center)},
    rounded corners=2pt]
    #1
  \end{tikzpicture}}

\tikzset{tens/.style={fill=tensorblue}}
\tikzset{prp/.style={fill=tensorgreen}}
\tikzset{msr/.style={fill=tensorred}}
\tikzset{tengrey/.style={fill=black!20}}

\usetikzlibrary{calc}
\newcommand{\wire}[4][]{
  \draw[#1] (#3,#2) -- (#4,#2)
}
\newcommand{\vwire}[4][]{
  \draw[#1] (#2,#3) -- (#2,#4)
}
\newcommand{\rect}[6][tens]{
  \draw[#1] (#4,#2) rectangle (#5,#3);
  \draw ($ 0.5*(#4,#2) + 0.5*(#5,#3) $) node {$#6$}
}
\renewcommand{\dot}[2]{
  \node at (#2,#1) [circle,fill,inner sep=1.5pt]{};
}
\renewcommand{\cross}[3][.25]{
  \draw (#3,#2) circle (#1) node {};
  \draw ($ (#3,#2) - (#1,0) $) -- ($ (#3,#2) + (#1,0) $) node{};
  \draw ($ (#3,#2) - (0,#1) $) -- ($ (#3,#2) + (0,#1) $) node{};
}
\newcommand{\cnot}[3]{
  \dot{#2}{#1};
  \vwire{#1}{#2}{#3};
  \cross{#3}{#1};
}

% measurement meter
\newcommand{\meter}[5][tengrey]{
  \rect[#1]{#2}{#3}{#4}{#5}{};
  \path[path picture={
    \draw[black] ([shift={(.1,.1)}]
    path picture bounding box.south west) to[bend left=50]
    ([shift={(-.1,.1)}]
    path picture bounding box.south east);
    \draw[black,-latex] ([shift={(0,.1)}]
    path picture bounding box.south)
    -- ([shift={(.2,-.1)}]path picture bounding box.north);}]
  (#4,#2) rectangle (#5,#3);
}


%%%%%%%%%%%%%%%%%%%%%%%%%%%%%%%%%%%%%%%%%%%%%%%%%%%%%%%%%%%%%%%%%%%%%%

\definecolor{green}{rgb}{0,0.6,0}
\definecolor{grey}{rgb}{0.35,0.35,0.35}

% for drawing arrows
\usetikzlibrary{arrows,shapes}
\tikzstyle{every picture}+=[remember picture]
\newcommand{\tikzmark}[1]{
  \tikz[remember picture] \node[coordinate] (#1) {#1};
}

% for uncovering under/over braces
\newcommand<>{\uncoverubrace}[2]{%
  \onslide#3 \underbrace{ \onslide<1->%
  #1%
  \onslide#3 }_{#2} \onslide<1->%
}
\newcommand<>{\uncoverobrace}[2]{%
  \onslide#3 \overbrace{ \onslide<1->%
  #1%
  \onslide#3 }^{#2} \onslide<1->%
}

%%%%%%%%%%%%%%%%%%%%%%%%%%%%%%%%%%%%%%%%%%%%%%%%%%%%%%%%%%%%%%%%%%%%%%

\title{Circuit cutting: theory and practice}%
\author{M. A. Perlin}%
\date{9 August 2019}


\begin{document}

\begingroup
\begin{frame}[plain]
  \titlepage
\end{frame}
\endgroup
\addtocounter{framenumber}{-1}

\begin{frame}
  \frametitle{A simple identity}

  Resolution of the identity
  \begin{align*}
    I =
    \begin{pmatrix}
      1 & 0 \\ 0 & 1
    \end{pmatrix}
    = \sum_{b\in\set{0,1}} \op{b}
  \end{align*}

  \vspace{1em}

  \uncover<2->{
    $N$-qubit state $\ket\psi$, individual qubit $j$
  }

  \vspace{1em} \uncover<3->{
    \hspace{.1\textwidth}
    {\color{red} projection}\makebox[0pt]{\tikzmark{MSR}} onto
    (sub-normalized) state of $\p{N-1}$ qubits
  }

  \vspace{.5em}
  \uncover<2->{
    \begin{align*}
      \ket\psi = I_j\ket\psi
      = \sum_b \uncoverubrace<4->
      {\color<4->{green}{\ket{b}_j}}{\tikzmark{prp}}
      \uncoverobrace<3->
      {\color<3->{red}{\prescript{}{j}{\bk{b|\psi}}}}
      {\tikzmark{msr}}
      \uncover<5->{
        = \sum_b
        {\color{red} \prescript{}{j}{\bk{b|\psi}}}
        {\color{green} \otimes_j\ket{b}}
      }
    \end{align*}
  }

  \uncover<3->{
    \tikz[overlay]{
      \path[thick,<->]
      ([xshift=-.1em,yshift=.05em]msr) edge [] ([yshift=-.5em]MSR);
    }
  }

  \uncover<4->{
    \vspace{-1em}
    \hspace{.05\textwidth}
    state {\color{green} preparation}\tikzmark{PRP}

    \tikz[overlay]{
      \path[thick,<->]
      ([xshift=-.1em]prp) edge [bend left] ([yshift=.25em]PRP);
    }
  }
\end{frame}

\begin{frame}
  \frametitle{A simple identity (with pictures)}
  \begin{align*}
    \ket\psi = \sum_{b\in\set{0,1}}
    {\color{red}{\prescript{}{j}{\bk{b|\psi}}}}
    {\color{green}{\otimes_j \ket{b}}}
  \end{align*}

  \begin{align*}
    \diagram{
      \wire{-1.5}{-.5}{3.5};
      \wire{0}{-.5}{3.5};
      \wire{1.5}{-.5}{3.5};
      \rect{-2}{2}{0}{3}{\t{circuit}};
      \rect[prp]{-.5}{.5}{-1.5}{-.5}{0};
      \rect[prp]{1}{2}{-1.5}{-.5}{0};
      \rect[prp]{-2}{-1}{-1.5}{-.5}{0};
    }
    ~ = ~
    \diagram{
      \wire{-1.5}{0}{2};
      \wire{0}{0}{2};
      \wire{1.5}{0}{2};
      \rect{-2}{2}{0}{1.5}{\psi};
    }
    \uncover<2->{
      ~ = ~
      \sum_{b\in\set{0,1}} ~
      \diagram{
        \wire{-1.5}{0}{5};
        \wire{0}{0}{2};
        \wire{1.5}{0}{5};
        \wire{0}{4.5}{5};
        \rect{-2}{2}{0}{1.5}{\psi};
        \rect[msr]{-.5}{.5}{2}{3}{b};
        \rect[prp]{-.5}{.5}{3.5}{4.5}{b};
      }
    }
  \end{align*}

  \vspace{.em}

  \uncover<3->{
    \begin{center}
      in practice: state {\color{red} projection} not allowed!
    \end{center}
  }
  \begin{enumerate}
  \item<4-> measure in some basis, e.g.~$\set{\ket{0},\ket{1}}$
  \item<5-> post-select on some measurement outcome, e.g.~$\ket{0}$
  \item<6-> complex amplitudes inaccessible
  \end{enumerate}
  \uncover<7->{
    \begin{align*}
      \ket\psi ~\to~ \rho = \op\psi
      &&
      \prescript{}{j}{\bk{0|\psi}} ~\to~
      \tr_j\p{\rho\op{0}_j}
    \end{align*}
  }
\end{frame}

\begin{frame}
  \frametitle{A new identity}

  \vspace{.5em}
  Multi-qubit state (density operator) $\rho$, Pauli operators $M$:
  \begin{align*}
    \rho = \f12 \sum_{M\in\set{X,Y,Z,I}}
    \uncoverubrace<2->{\tr_j\p{\rho M_j}}{\tikzmark{msr}}
    \uncoverubrace<2->{\otimes_j M}{\tikzmark{prp}}
  \end{align*}

  \uncover<2->{
    \vspace{.5em}
    measurement and post-selection? \tikzmark{MSR}

    \vspace{1em}
    \hspace{0.3\linewidth} state preparation? \tikzmark{PRP}

    \tikz[overlay]{
      \path[thick,<->]
      (msr) edge [bend left] ([yshift=.25em]MSR);
      \path[thick,<->]
      (prp) edge [bend left] ([yshift=.5em]PRP);
    }
  }

  \uncover<3->{
    Expand $M$ in diagonal basis, e.g.~$Z=\op{0}-\op{1}$
  }
  \uncover<4->{
    \begin{align*}
      M = \sum_{s\in\set{\pm1}} c_{Ms} \op{M,s}
    \end{align*}
  }
  \uncover<5->{
    \vspace{-1em}
    \begin{align*}
      \rho = \f12
      \sum_{M,r,s} {\color{grey}{c_{Mr} c_{Ms}}} ~
      {\color{red} \tr_j\p{\rho \tikzmark{state1} \op{M,r}_j}}
      {\color{green} \otimes_j \tikzmark{state2} \op{M,s}}
      && ~
    \end{align*}

    \vspace{-1.5em}
    \hspace{0.8\linewidth} \tikzmark{STATE} states!

    \tikz[overlay]{
      \path[thick,->]
      ([yshift=-.5em,xshift=1.7em]state1) edge [bend right] (STATE);
      \path[thick,->]
      ([yshift=-.5em,xshift=1.7em]state2) edge [bend right]
      ([yshift=.4em]STATE);
    }
  }
\end{frame}

\begin{frame}
  \frametitle{The general prescription}

  \vspace{1em}
  Multi-qubit state $\rho$, single-qubit states $\sigma,\tau$,
  coefficients $c_{\sigma\tau}$:
  \begin{align*}
    \rho = \sum_{\sigma,\tau} c_{\sigma\tau}
    \uncoverubrace<2->{\tr_j\p{\rho \sigma_j}}{\tikzmark{msr}}
    \uncoverubrace<2->{\otimes_j \tau}{\tikzmark{prp}}
  \end{align*}

  \uncover<2->{
    measure qubit $j$, find state $\sigma$ \tikzmark{MSR}

    \vspace{0.5em}
    prepare state $\tau$ on qubit $j$ \tikzmark{PRP}

    \tikz[overlay]{
      \path[thick,<->]
      (msr) edge [bend left] ([yshift=.25em]MSR);
      \path[thick,<->]
      (prp) edge [bend left] ([yshift=.5em]PRP);
    }
  }

  \vspace{-.5em}
  \uncover<3->{
    \begin{align*}
      \diagram{
        \wire{0}{0}{2};
        \wire{1.5}{0}{2};
        \wire{-1.5}{0}{2};
        \uncover<4->{
          \wire{0}{0}{6};
          \wire{1.5}{0}{6};
          \wire{-1.5}{0}{6};
        }
        \rect{-2}{2}{0}{1.5}{\rho};
        \uncover<3>{
          \meter{-.5}{.5}{2}{3};
          \meter{1}{2}{2}{3};
          \meter{-2}{-1}{2}{3};
        }
        \uncover<4->{
          \rect{-2}{2}{2.5}{5.5}{\t{circuit}};
          \meter{-.5}{.5}{6}{7};
          \meter{1}{2}{6}{7};
          \meter{-2}{-1}{6}{7};
        }
      }
      ~ = ~
      \sum_{\sigma,\tau} c_{\sigma\tau} ~
      \diagram{
        \wire{0}{0}{2};
        \wire{0}{5}{5.5};
        \wire{-1.5}{0}{5.5};
        \wire{1.5}{0}{5.5};
        \uncover<4->{
          \wire{-1.5}{0}{9};
          \wire{1.5}{0}{9};
          \wire{0}{5}{9};
        }
        \rect{-2}{2}{0}{1.5}{\rho};
        \rect[msr]{-.5}{.5}{2}{3.25}{\sigma};
        \rect[prp]{-.5}{.5}{3.75}{5}{\tau};
        \uncover<3>{
          \meter{-.5}{.5}{5.5}{6.5};
          \meter{1}{2}{5.5}{6.5};
          \meter{-2}{-1}{5.5}{6.5};
        }
        \uncover<4->{
          \rect{-2}{2}{5.5}{8.5}{\t{circuit}};
          \meter{-.5}{.5}{9}{10};
          \meter{1}{2}{9}{10};
          \meter{-2}{-1}{9}{10};
        }
      }
    \end{align*}
  }

  \uncover<5->{
    {\it Anywhere} in a circuit:
    \begin{align*}
      \diagram{
        \wire{0}{0}{3};
      }
      ~ = ~
      \sum_{\sigma,\tau} c_{\sigma\tau} ~
      \diagram{
        \wire{.5}{-.5}{0};
        \wire{.5}{2.5}{3};
        \rect[msr]{0}{1}{0}{1}{\sigma};
        \rect[prp]{0}{1}{1.5}{2.5}{\tau};
      }
    \end{align*}
  }
\end{frame}

\begin{frame}
  \frametitle{Cutting a GHZ circuit}
  \vspace{-.25em}
  \begin{align*}
    \rho_{\t{GHZ}} = ~
    \diagram{
      \wire{0}{-.5}{3.75};
      \wire{1.5}{-.5}{3.75};
      \wire{-1.5}{-.5}{3.75};
      \rect[prp]{-.5}{.5}{-1.5}{-.5}{0};
      \rect[prp]{1}{2}{-1.5}{-.5}{0};
      \rect[prp]{-2}{-1}{-1.5}{-.5}{0};
      \rect{1}{2}{0}{1}{H};
      \cnot{1.75}{1.5}{0};
      \cnot{2.75}{0}{-1.5};
    }
    \uncover<2->{
      ~ = ~
      \sum_{\sigma,\tau} c_{\sigma\tau} ~
      \diagram{
        \wire{0}{-.5}{2.5};
        \wire{0}{5}{6.5};
        \wire{1.5}{-.5}{6.5};
        \wire{-1.5}{-.5}{6.5};
        \rect[prp]{-.5}{.5}{-1.5}{-.5}{0};
        \rect[prp]{1}{2}{-1.5}{-.5}{0};
        \rect[prp]{-2}{-1}{-1.5}{-.5}{0};
        \rect{1}{2}{0}{1}{H};
        \cnot{1.75}{1.5}{0};
        \rect[msr]{-.5}{.5}{2.5}{3.5}{\sigma};
        \rect[prp]{-.5}{.5}{4}{5}{\tau};
        \cnot{5.5}{0}{-1.5};
      }
    }
  \end{align*}

  \vspace{1em}
  \uncover<3->{
    Circuit fragments -- smaller than original circuit!
    \begin{align*}
      \rho_1\p{\sigma} = ~
      \diagram{
        \wire{0}{-.5}{3};
        \wire{1.5}{-.5}{3};
        \rect[prp]{-.5}{.5}{-1.5}{-.5}{0};
        \rect[prp]{1}{2}{-1.5}{-.5}{0};
        \rect{1}{2}{0}{1}{H};
        \cnot{1.75}{1.5}{0};
        \rect[msr]{-.5}{.5}{2.5}{3.5}{\sigma};
      }
      &&
      \t{and}
      &&
      \rho_2\p{\tau} = ~
      \diagram{
        \wire{0}{-.5}{1.5};
        \wire{1.5}{-.5}{1.5};
        \rect[prp]{-.5}{.5}{-1.5}{-.5}{0};
        \rect[prp]{1}{2}{-1.5}{-.5}{\tau};
        \cnot{0.5}{1.5}{0};
      }
    \end{align*}
  }

  \uncover<4->{
    Reconstruction:
    $\rho_{\t{GHZ}} = \displaystyle\sum_{\sigma,\tau} c_{\sigma\tau} ~
    \rho_1\p{\sigma} \otimes \rho_2\p{\tau}$
    \vspace{-.5em}
    \begin{align*}
      \diagram{
        \wire{0}{-.5}{3.75};
        \wire{1.5}{-.5}{3.75};
        \wire{-1.5}{-.5}{3.75};
        \rect[prp]{-.5}{.5}{-1.5}{-.5}{0};
        \rect[prp]{1}{2}{-1.5}{-.5}{0};
        \rect[prp]{-2}{-1}{-1.5}{-.5}{0};
        \rect{1}{2}{0}{1}{H};
        \cnot{1.75}{1.5}{0};
        \cnot{2.75}{0}{-1.5};
        \meter{-.5}{.5}{3.5}{4.5};
        \meter{1}{2}{3.5}{4.5};
        \meter{-2}{-1}{3.5}{4.5};
      }
      ~ = ~
      \sum_{\sigma,\tau} c_{\sigma\tau} ~
      \diagram{
        \wire{0}{-.5}{3};
        \wire{1.5}{-.5}{3};
        \rect[prp]{-.5}{.5}{-1.5}{-.5}{0};
        \rect[prp]{1}{2}{-1.5}{-.5}{0};
        \rect{1}{2}{0}{1}{H};
        \cnot{1.75}{1.5}{0};
        \rect[msr]{-.5}{.5}{2.5}{3.5}{\sigma};
        \meter{1}{2}{2.5}{3.5};
      }
      ~ \otimes ~
      \diagram{
        \wire{0}{-.5}{1.5};
        \wire{1.5}{-.5}{1.5};
        \rect[prp]{-.5}{.5}{-1.5}{-.5}{0};
        \rect[prp]{1}{2}{-1.5}{-.5}{\tau};
        \cnot{0.5}{1.5}{0};
        \meter{-.5}{.5}{1.5}{2.5};
        \meter{1}{2}{1.5}{2.5};
      }
    \end{align*}
  }
\end{frame}

\begin{frame}
  \frametitle{Improvements (teaser)}
  \vspace{1em}
  Reconstruction with one cut:
  \begin{align*}
    \rho = \sum_{\sigma,\tau} c_{\sigma\tau} ~
    \rho_1\p{\sigma} \otimes \rho_2\p{\tau}
  \end{align*}

  \uncover<2->{
    Old method: $\sigma,\tau\in\set{\op{M,s}}$ for Pauli ops $M$ and
    $s\in\set{\pm1}$
  }

  \vspace{.5em}
  \uncover<3->{
    Loop over 8 states $\op{M,s}$
    \vspace{.5em}
    \begin{itemize}
    \item total of $8^2=64$ terms \hfill

      \vspace{.25em}
      \begin{itemize}
      \item<4-> $8^{2C}$ terms for a circuit with $C$ cuts
      \end{itemize}

      \vspace{.5em}
    \item need $\rho_n\p{\sigma}$ for 6 {\it unique} states $\sigma$
      (eigenstates of $X,Y,Z$)

      \vspace{.25em}
      \begin{itemize}
      \item<4-> $6^{c_n}$ states for fragment $n$ with $c_n$ cuts
    \end{itemize}
    \end{itemize}
  }

  \vspace{.5em}
  \uncover<5->{
    New method:
  }
  \begin{itemize}
    \vspace{.5em}
  \item<5-> knowing $\rho_n\p{\sigma}$ for 4 states $\sigma$
    sufficient to know {\it any} $\rho_n\p{M}$
    \vspace{.5em}
  \item<6-> using
    $\rho = \f12 \displaystyle\sum_{M\in\set{X,Y,Z,I}} \rho_1\p{M}
    \otimes \rho_2\p{M}$ directly $\to$ {\color<7->{red}{$4^C$} terms}
  \end{itemize}
  \vspace{-1em}
  ~\hfill \uncover<7->{\color{red} sampling!}
\end{frame}

\end{document}
