\documentclass[nofootinbib,notitlepage,11pt]{revtex4-2}

%%% linking references
\usepackage{hyperref}
\hypersetup{
  breaklinks=true,
  colorlinks=true,
  linkcolor=blue,
  filecolor=magenta,
  urlcolor=cyan,
}

%%% header / footer
\usepackage{fancyhdr} % easier header and footer management
\pagestyle{fancy} % page formatting style
\fancyhf{} % clear all header and footer text
\renewcommand{\headrulewidth}{0pt} % remove horizontal line in header
\usepackage{lastpage} % for referencing last page
\cfoot{\thepage~of \pageref{LastPage}} % "x of y" page labeling
\usepackage[yyyymmdd]{datetime} % set date format
\renewcommand{\dateseparator}{.} % separate dates by a dot

%%% symbols, notations, etc.
\usepackage{physics,braket,bm,amssymb} % physics and math
\usepackage{mathtools} % for floor delimiter

\renewcommand{\t}{\text} % text in math mode
\newcommand{\f}[2]{\dfrac{#1}{#2}} % shorthand for fractions
\newcommand{\p}[1]{\left(#1\right)} % parenthesis
\renewcommand{\sp}[1]{\left[#1\right]} % square parenthesis
\renewcommand{\set}[1]{\left\{#1\right\}} % curly parenthesis
\newcommand{\bk}{\Braket} % shorthand for braket notation
\renewcommand{\v}{\bm} % bold vectors
\newcommand{\uv}[1]{\bm{\hat{#1}}} % unit vectors
\renewcommand{\c}{\cdot} % inner product

\newcommand{\B}{\mathcal{B}}
\renewcommand{\O}{\mathcal{O}}
\newcommand{\Z}{\mathbb{Z}}
\newcommand{\diag}{\mathrm{diag}\,}

\DeclarePairedDelimiter\floor{\lfloor}{\rfloor}
\renewcommand*{\thefootnote}{\alph{footnote}}


\begin{document}
\count\footins = 1000 % play nicely with long footnotes

\title{Cutting and stitching quantum circuits}
\author{Michael A. Perlin}
\date{\today}

\maketitle
\thispagestyle{fancy}

\tableofcontents


\section{Introduction}
\label{sec:intro}

Every quantum computation can be represented by a quantum circuit.
This quantum circuit in turn defines a probability distribution over
the possible (classical) states of its qubits.  The purpose of
performing a quantum computation is essentially to sample the
distribution defined by a corresponding quantum circuit.  Some quantum
circuits, however, are too large to run directly on any available
hardware.  To address this problem, the authors of some recent
work\cite{peng2019simulating} developed a scheme to (i) ``cut'' large
quantum circuits into smaller {\it sub-circuits} (or {\it fragments})
that can executed on available hardware, and (ii) reconstruct the
probability distribution defined by a large quantum circuit from
probability distributions associated with its fragments.

In these notes, we present our own take on the main result in
ref.~[\citenum{peng2019simulating}], introducing a new formalism for
thinking about circuit cutting, fragments, and reconstruction.  In
addition to clarifying the equivalence between two cutting strategies
used in the circuit cutting groups at ANL, our formalism motivates a
few simple and natural improvements to the cutting scheme in
ref.~[\citenum{peng2019simulating}], yielding major reductions to the
runtimes and memory footprints necessary to simulate fragments and
reconstruct probability distributions.  After presenting these
improvements, we discuss the problem of {\it sampling} (as opposed to
reconstructing) a probability distribution defined by large circuits,
and suggest some potential resolutions to this problem.

We will assume that the reader is familiar with tensor network
representations of quantum circuits.  For brevity, we will generally
blur the distinction between a tensor network, the circuit it
represents (when applicable), and the state prepared by this circuit.
We will also blur the distinction between the dangling edges of a
tensor network and qubit degrees of freedom associated with these
dangling edges.


\section{The general cut-and-stitch prescription}
\label{sec:networks}

The cutting scheme in ref.~[\citenum{peng2019simulating}] is based on
a simple identity for linear operators $\rho$ on the Hilbert space of
a collection of qubits:
\begin{align}
  \rho \simeq \f12 \sum_{M\in\B} \tr_j\p{\rho M_j}\otimes M,
  \label{eq:core_identity}
\end{align}
where $\simeq$ denotes equality up to a permutation of tensor factors;
$\B\equiv\set{I,X,Y,Z}$ is the set of singe-qubit identity and Pauli
operators; $\tr_j$ denotes the partial trace with respect to qubit
$j$; and $M_j$ denotes an operator which acts with $M$ on qubit $j$
and trivially (i.e.~with the identity $I$) on all other qubits.  The
identity in \eqref{eq:core_identity} follows trivially from the fact
that $\B/\sqrt2$ forms an orthonormal basis for the vector space of
single-qubit operators equipped with the trace inner product
$\bk{A,B}\equiv\tr\p{A^\dag B}$.  This identity is a direct analogue
(e.g.~for density operators) of the pure-state identity
\begin{align}
  \ket\psi = I_j\ket\psi=\sum_{b\in\set{0,1}} \op{b}_j \ket\psi.
\end{align}

When a circuit is represented by a tensor network, the identity in
\eqref{eq:core_identity} can (if the network structure allows) be used
to ``cut'' the network into two disjoint sub-networks.  Just as the
entire tensor network defines an operator $\rho$ on the Hilbert space
of its dangling edges (e.g.~a density operator on the output qubits of
a circuit), these sub-networks define operators on the Hilbert space
of their respective dangling edges.  Written in terms of the operators
defined by these sub-networks, the identity in
\eqref{eq:core_identity} becomes
\begin{align}
  \rho \simeq \f12 \sum_{M\in\B} \rho_1\p{M} \otimes \rho_2\p{M},
  \label{eq:cut_identity}
\end{align}
where $\rho_n\p{M}$ is the operator defined by sub-network $n$ when
attaching the operator $M$ at the location of the cut.  Note that if
$\rho_n\p{M}$ is an $Q_n$-qubit operator, then $\rho_n$ is a linear
map from the space of single-qubit operators to the space of
$Q_n$-qubit operators.  The map $\rho_n$ is naturally represented by a
tensor network with $Q_n+1$ dangling edges, which maps single-qubit
operators on the dangling {\it cut edge} $c_n$ of $\rho_n$ to
$Q_n$-qubit operators on the dangling {\it non-cut edges}
$\bar{\v c}_n$ of $\rho_n$.  The operator $\rho_n\p{M}$ is recovered
from $\rho_n$ by contracting the dangling edge $c_n$ with the vector
$M$.

More generally, we can make $C$ cuts that split a tensor network
$\rho$ into $N$ sub-networks $\set{\rho_n}$.  In order to stitch all
sub-networks together, we will now need to sum over a list of
operators $\v M\equiv\p{M_1,M_2,\cdots,M_C}\in\B^C$, where the
operator $M_k$ is ``assigned'' to stitch (or cut) $k$.  Denoting the
set of all stitches incident to sub-network $\rho_n$ by $\v s_n$, we
define $\floor{\v M}_n\equiv\p{M_s:s\in\v s_n}$ to be the restriction
of $\v M$ to the stitches in $\v s_n$; that is, $\floor{\v M}_n$ is a
list of the operators in $\v M$ that are associated with stitches
incident to $\rho_n$.  We can then expand the operator $\rho$ defined
by the entire network as
\begin{align}
  \rho \simeq \f1{2^C} \sum_{\v M\in\B^C}
  \bigotimes_{n\in\Z_N} \rho_n\p{\floor{\v M}_n},
  \label{eq:general_identity}
\end{align}
where $\rho_n\p{\floor{\v M}_n}$ is acquired by contracting the cut
edges $\v c_n$ of sub-network $\rho_n$ with the vectors in
$\floor{\v M}_n$.  Although we will keep \eqref{eq:general_identity}
in mind as the general prescription for reconstructing a circuit
$\rho$ from its fragments $\set{\rho_n}$, for simplicity we will
generally consider the single-cut case in \eqref{eq:cut_identity}
throughout this work.


\section{Application to quantum circuits and runtime improvements}
\label{sec:circuits}

The expansions in \eqref{eq:cut_identity} and
\eqref{eq:general_identity} provides a prescription to cut and stitch
tensor networks, but what about quantum circuits?  What does it mean
to attach some operator (e.g.~$M\in\B$) to a circuit?  We will first
consider the attachment of a single-qubit density operator $\sigma$
rather than an arbitrary single-qubit operator $M$.  Once we
understand how to attach single-qubit density operators to circuits,
we can use the fact that these operators span the entire space of
single-qubit operators to build any operator we wish.

Cutting a circuit $\rho$ into two fragments $\rho_1$ and $\rho_2$
leaves a dangling ``output'' cut edge $c_1$ on fragment $\rho_1$ and a
dangling ``input'' cut edge $c_2$ on fragment $\rho_2$.  Attaching a
density operator $\sigma$ to the new dangling input edge $c_2$ of
fragment $\rho_2$ has a straightforward interpretation: we simply
prepare the state $\sigma$ at edge $c_2$.  In this way, the operator
$\rho_2$ can be viewed as a {\it conditional state}, where the
``condition'' $\sigma$ (i.e.~preparing the state $\sigma$ at edge
$c_2$) induces the state $\rho_2\p{\sigma}$ on the remaining dangling
edges (qubits) $\bar{\v c}_2$ of $\rho_2$.\footnote{The preparation of
  a state $\sigma$ at edge $c_2$ is essentially the ``new method'' in
  ref.~[\citenum{peng2019simulating}] for preparing
  $\rho_2\p{\sigma}$.  The ``old method'' for preparing
  $\rho_2\p{\sigma}$ was to initialize a pair of qubits in the
  maximally entangled state $\p{\ket{00}+\ket{11}}/\sqrt2$, insert one
  of these qubits at $c_2$, and then post-select all results on
  measurement of the second qubit in a desired pure state $\sigma$.
  Such post-selection projects onto an initial product state
  $\sigma\otimes\sigma$ of the maximally entangled qubits, and is
  therefore equivalent to simply preparing the first qubit in $\sigma$
  and forgetting about the second qubit entirely, as in the ``new
  method''.}

In a similar manner, $\rho_1$ is a conditional operator for which the
condition $\sigma$ corresponds to {\it finding} the qubit at the
``output'' cut edge $c_1$ of $\rho_1$ in the state $\sigma$.  Unlike
with the ``input'' case of $\rho_2\p{\sigma}$, however,
$\rho_1\p{\sigma}$ is only {\it proportional} to a state prepared on
the remaining edges $\bar{\v c}_1$ of circuit $\rho_1$ under condition
$\sigma$; the proportionality factor $\tr\rho_1\p{\sigma}$ is equal to
the probability of finding the qubit at edge $c_1$ in the state
$\sigma$.  More specifically, with slight abuse of notation we can say
that $\rho_1\p{\sigma}=\tr_{c_1}\p{\rho_1\sigma_{c_1}}$.  If $\rho_1$
is a pure state, for example, then we can expand $\rho_1 = \op\psi$
with
$\ket\psi = \sum_{b\in\set{0,1}} \psi_b \ket{\psi_b}_{\bar{\v c}_1}
\otimes \ket{b}_{c_1}$, where each $\ket{\psi_b}_{\bar{\v c}_1}$ is
some pure state on the non-cut-edges $\bar{\v c}_1$ of circuit
$\rho_1$.  The conditional operator $\rho_1\p{\sigma}$ for
e.g.~$\sigma=\op{0}$ is then
$\rho_1\p{\op{0}}=\abs{\psi_0}^2\op{\psi_0}_{\bar{\v c}_1}$.

We ultimately wish to sample the probability distribution
$p\equiv\diag\rho$ defined by the large quantum circuit $\rho$ in some
fixed computational basis. To do so, we first define the {\it
  conditional distributions} $p_n$ with
$p_n\p{M}\equiv\diag\rho_n\p{M}$.  We then observe that for any
density operator $\sigma$, the distribution $p_n\p{\sigma}$ is
straightforward to determine by simulating the fragment $\rho_n$.  As
$\rho_2\p{\sigma}$ is an ordinary quantum state on edges (qubits)
$\bar{\v c}_2$ of fragment $\rho_2$, the distribution
$p_2\p{\sigma}=\diag\rho_2\p{\sigma}$ is simply a probability
distribution over measurement outcomes on the qubits at
$\bar{\v c}_2$.  The distribution
$p_1\p{\sigma}=\diag\rho_1\p{\sigma}$ is similarly the probability
distribution over measurement outcomes on the non-cut-edges
$\bar{\v c}_1$ of fragment $\rho_1$ multiplied by the probability of
finding the qubit at edge $c_1$ in the state $\sigma$.  Once we know
the distributions $p_n\p{\sigma}$ for any state $\sigma$, we can build
any other distribution $p_n\p{M}$ by linearity of the conditionals
$p_n$ and a decomposition of $M$ into a linear combination of density
operators, e.g.~$p_n\p{Z}=p_n\p{\op{1}}-p_n\p{\op{0}}$.  In this way,
we reconstruct the probability distribution $p$ from conditionals
$p_n$ by restriction of \eqref{eq:cut_identity} to diagonal entries:
\begin{align}
  p \simeq \f12 \sum_{M\in\B} p_1\p{M} \otimes p_2\p{M}.
  \label{eq:cut_identity_probs}
\end{align}
At this point, for reasons that escape me the authors of
ref.~[\citenum{peng2019simulating}] decided to explicitly expand
$p_2\p{M}$ in a diagonal basis of $M$ while leaving $p_1\p{M}$ intact,
decomposing the overall probability distribution into eight terms as
\begin{align}
  p \simeq \f12 \sum_{\substack{B\in\set{X,Y,Z}\\s\in\set{\pm1}}}
  s\, p_1\p{B} \otimes p_2\p{B_s}
  + \f12\sum_{s\in\set{\pm1}} p_1\p{I} \otimes p_2\p{Z_s},
  \label{eq:cut_identity_probs_bad}
\end{align}
where $B_s\equiv\op{B_s}$ is the density operator corresponding to the
pure state $\ket{B_s}$ that is the eigenvector of a Pauli operator
$B\in\set{X,Y,Z}$ with eigenvalue $s$, i.e.
\begin{align}
  \ket{Z_+} = \ket0, && \ket{Z_-} = \ket1, &&
  \ket{X_\pm} = \f{\ket0 \pm \ket1}{\sqrt2}, &&
  \ket{Y_\pm} = \f{\ket0 \pm i\ket1}{\sqrt2}.
\end{align}
In practice, if the distributions $p_n\p{M}$ have a large memory
footprint, then the bottleneck (time-wise) for reconstructing $p$ is
the computation tensor products such as $p_1\p{M} \otimes p_2\p{M}$.
In this case, it is therefore much faster to first compute the
distributions $p_n\p{M}$ for all $M\in\B$, and then combine these
distributions according to \eqref{eq:cut_identity_probs}.  When
reconstructing the probability distribution of a circuit with $C$
cuts, using \eqref{eq:cut_identity_probs} in place of
\eqref{eq:cut_identity_probs_bad} reduces the number of tensor
products that one must compute from $8^C$ to $4^C$, which speeds up
reconstruction time by a factor of $\O\p{2^C}$.


\section{Informational completeness and memory improvements}
\label{sec:completeness}

Using a circuit fragment $\rho_n$ with a single cut edge $e_n$ for
reconstruction according to \eqref{eq:cut_identity_probs} requires
characterizing the conditional distribution $p_n$ for several states
(conditions) $\sigma$.  Once we know $p_n\p{Z_+}$ and $p_n\p{Z_-}$,
for example, we can compute both $p_n\p{Z}=p_n\p{Z_+}-p_n\p{Z_-}$ and
$p_n\p{I}=p_n\p{Z_+}+p_n\p{Z_-}$.  How many states $\sigma$ do we need
to choose, or equivalently how many distributions $p_n\p{\sigma}$ do
we need to compute, in order to determine all distribution $p_n\p{M}$
that we need for reconstruction?  The recombination prescription in
\eqref{eq:cut_identity_probs_bad} uses six states: $B_s$ for each of
$s\in\set{\pm1}$ and $B\in\set{X,Y,Z}$.  If a fragment $\rho_n$ has
$s_n$ incident stitches, therefore, at face value recombination
prescription in \eqref{eq:cut_identity_probs_bad} requires
characterizing $6^{s_n}$ real-valued distributions associated with
fragment $\rho_n$.  In a general sense, the reason that the six states
$\set{B_s}$ are sufficient to characterize any conditional is that
they form a complete basis for the space of single-qubit operators; in
fact, these states form an {\it over}-complete basis, as only four of
them are linearly independent.  We therefore do not need to keep track
of distributions $p_n\p{\sigma}$ for all six states in $\set{B_s}$, as
we can use the fact that $I=B_++B_-$ for any $B\in\set{X,Y,Z}$ to
recover, for example,
\begin{align}
  p_n\p{X_-}
  = p_n\p{I} - p_n\p{X_+}
  = p_n\p{Z_+} + p_n\p{Z_-} - p_n\p{X_+}.
  \label{eq:X-_expansion}
\end{align}
We thus only need to characterize a conditional $p_n$ with an {\it
  informationally complete} set of conditions $\sigma$ that
collectively span the space of single-qubit operators; this space is
four-dimensional, which means that we should only need to characterize
$4^{s_n}$ real-valued distributions for a fragment $\rho_n$ with $s_n$
incident stitches.

In the case of ``input'' conditions, as for the conditional $p_2$ in
\eqref{eq:cut_identity_probs}, the expansion in
\eqref{eq:X-_expansion} makes it clear that it suffices to prepare
each state $\sigma \in \set{X_+,Y_+,Z_+,Z_-}$, and store the
corresponding distributions $p_2\p{\sigma}$.  This choice of basis for
single-qubit operators, however, is ``biased'' along the $z$-axis,
which loosely speaking introduces unevenly distributed error in any
empirical characterization of the conditional $p_n$.  An unbiased
basis of states is provided by any {\it symmetric, informationally
  complete, positive operator-valued measure} (SIC-POVM), which for a
qubit consists of the states on the four corners of a regular
tetrahedron inscribed in the Bloch sphere.  The choice of orientation
for the tetrahedron is arbitrary; one such choice consists of unit
vectors $\uv v_j$ proportional to
\begin{align}
  \v v_1 \equiv \p{ +1, +1, +1 }, &&
  \v v_2 \equiv \p{ +1, -1, -1 }, &&
  \v v_3 \equiv \p{ -1, +1, -1 }, &&
  \v v_4 \equiv \p{ -1, -1, +1 },
\end{align}
which define states $S_j\equiv\p{I+\uv v_j\c\v X}/2$ with
$\v X\equiv\p{X,Y,Z}$.  Knowing the distribution $p_n\p{\sigma}$ for
all states $\sigma\in\set{S_j}$ allows us to compute $p_n\p{M}$ for
any single-qubit operator $M=M_0I+\v M\c\v X$ via the expansion
\begin{align}
  M = \f12 \sum_j \p{ M_0 + 3 \v M\c\uv v_j } S_j
  &&
  \implies
  &&
  p_n\p{M} = \f12 \sum_j \p{ M_0 + 3 \v M\c\uv v_j } p_n\p{S_j}.
  \label{eq:SIC_expansion}
\end{align}

In the case of ``output'' conditions, as for the conditional $p_1$ in
\eqref{eq:cut_identity_probs}, ideally we would perform a SIC-POVM
measurement directly.  While there has been some
theoretical\cite{tabia2012experimental} and
experimental\cite{bent2015experimental, bian2015realization} progress
in implementing direct SIC-POVM measurements, however, for now we are
limited to making projective measurements of qubits in some basis of
two orthogonal states.  Making projective measurements of SIC-POVM
basis elements would therefore require measuring the qubit at edge
$e_1$ of fragment $\rho_1$ in four different bases, which can yield
one of eight different outcomes.  Four of these outcomes prepare a
SIC-POVM state $S_j$, which in turn prepares the state (proportional
to) $\rho_1\p{S_j}$ at the non-cut edges $\bar{\v e}_1$ of fragment
$\rho_1$; measuring the qubits at $\bar{\v e}_1$ then samples the
distribution $p_1\p{S_j}$.  The problem with this measurement
strategy, however, is that the remaining possible measurement
outcomes, $\set{I-S_j}$, do not give us any (direct) information about
the four distributions $p_1\p{S_j}$ that we are trying to
characterize.  We must therefore either store {\it eight}
distributions per ``output'' condition, or throw out half of our
possible measurement outcomes.

The resolution to this dilemma is simple: we simply go back to
measuring in the three eigenbases of Pauli operators $X$, $Y$, $Z$.
Rather than keeping track of one distribution for each of six possible
measurement outcomes, however, we can (i) keep track of the
distributions $p_1\p{B_+}$ for each $B\in\set{X,Y,Z}$ as appropriate,
i.e.~add to a histogram empirically characterizing $p_1\p{B_+}$ after
finding the measured qubit at output edge $e_1$ in the state $B_+$,
and then (ii) {\it always} collect data on the distribution $p_1\p{I}$
{\it regardless} of the measurement outcome, as
\begin{align}
  I = \f13 \sum_{\substack{B\in\set{X,Y,Z}\\s\in\set{\pm1}}} B_s
  &&
  \implies
  &&
  p_n\p{I} = \f13 \sum_{\substack{B\in\set{X,Y,Z}\\s\in\set{\pm1}}}
  p_n\p{B_s}.
\end{align}
This measurement scheme is economical in the sense that it (i) stores
four probability distributions per ``output'' condition (i.e.~the
minimum number necessary for reconstruction), and (ii) makes use of
all possible measurement outcomes.\footnote{Note that making
  projective measurements in only two different bases is insufficient
  to characterize the conditional distribution $p_1$.  Although
  measurements in two different bases yield four possible outcomes,
  these outcomes are not linearly independent, as can be seen from
  \eqref{eq:X-_expansion}.}  Knowing the distribution $p_n\p{\sigma}$
for all conditions $\sigma\in\set{I,X_+,Y_+,Z_+}$ allows us to compute
$p_n\p{M}$ for any single-qubit operator $M = M_0 I + \v M\c\v X$ with
$\v M\equiv\p{M_X,M_Y,M_Z}$ via the expansion
\begin{align}
  M = 2 \v M\c\v X_+ + \p{ M_0 - \sum_{B\in\set{X,Y,Z}} M_B } I
\end{align}
where $\v X_+\equiv\p{X_+,Y_+,Z_+}$ and (for reference)
$B_+=\p{I+B}/2$.


\section{A quick aside: complex-valued (amplitude) conditionals}
\label{sec:amplitudes}

If we can simulate a fragment directly (i.e.~via ``statevector''
simulations) on a classical computer and store complex amplitudes for
all possible measurement outcomes, then instead
\eqref{eq:core_identity} we can cut a circuit using the simple
pure-state identity
\begin{align}
  \ket\psi = \sum_{b\in\set{0,1}} \op{b}_j \ket\psi,
\end{align}
where $j$ indexes the qubit at the cut edge of the circuit.  The
analogue of \eqref{eq:cut_identity} for stitching together two
fragments of a circuit with one cut becomes
\begin{align}
  \ket\psi \simeq
  \sum_{b\in\set{0,1}} \ket{\psi_1\p{b}} \otimes \ket{\psi_2\p{b}}.
\end{align}
Although reconstruction now requires assigning one of two values
$b\in\set{0,1}$ to each stitch, as opposed to one of four operators
$M\in\B$, the memory footprint of each fragment $\psi_n$ is the same
as before because the conditional distributions $\psi_n$ are now
complex-valued (as opposed to real-valued).  Nonetheless, direct
simulations of a fragment yield the distributions $\psi_n\p{b}$ for
$b\in\set{0,1}$ directly; the conditionals $\rho_n$, in contrast, have
to be characterized for all $M\in\B$ by post-processing simulation
results.  When performing direct simulations of fragments, therefore,
computing and recombining complex-valued conditionals $\psi_n$ rather
than real-valued conditionals $\rho_n$ is faster due to reduced
computational overhead.


\section{The sampling problem}
\label{sec:sampling}

-- TO BE CONTINUED --


\bibliography{\jobname.bib}

\end{document}
