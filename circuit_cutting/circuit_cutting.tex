\documentclass[nofootinbib,notitlepage,11pt]{revtex4-2}

%%% linking references
\usepackage{hyperref}
\hypersetup{
  breaklinks=true,
  colorlinks=true,
  linkcolor=blue,
  filecolor=magenta,
  urlcolor=cyan,
}

%%% header / footer
\usepackage{fancyhdr} % easier header and footer management
\pagestyle{fancy} % page formatting style
\fancyhf{} % clear all header and footer text
\renewcommand{\headrulewidth}{0pt} % remove horizontal line in header
\usepackage{lastpage} % for referencing last page
\cfoot{\thepage~of \pageref{LastPage}} % "x of y" page labeling
\usepackage[yyyymmdd]{datetime} % set date format
\renewcommand{\dateseparator}{.} % separate dates by a dot

%%% symbols, notations, etc.
\usepackage{physics,braket,bm,amssymb} % physics and math
\usepackage{mathtools} % for floor delimiter and \prescript command

\renewcommand{\t}{\text} % text in math mode
\newcommand{\f}[2]{\dfrac{#1}{#2}} % shorthand for fractions
\newcommand{\p}[1]{\left(#1\right)} % parenthesis
\renewcommand{\sp}[1]{\left[#1\right]} % square parenthesis
\renewcommand{\set}[1]{\left\{#1\right\}} % curly parenthesis
\newcommand{\bk}{\Braket} % shorthand for braket notation
\renewcommand{\v}{\bm} % bold vectors
\newcommand{\uv}[1]{\bm{\hat{#1}}} % unit vectors
\renewcommand{\c}{\cdot} % inner product

\newcommand{\B}{\mathcal{B}}
\renewcommand{\O}{\mathcal{O}}
\newcommand{\Z}{\mathbb{Z}}
\newcommand{\diag}{\mathrm{diag}\,}


\usepackage[inline]{enumitem}
\setlist[enumerate,1]{label={(\roman*)}}
\renewcommand*{\thefootnote}{\alph{footnote}}
\DeclarePairedDelimiter\floor{\lfloor}{\rfloor}


%%%%%%%%%%%%%%%%%%%%%%%%%%%%%%%%%%%%%%%%%%%%%%%%%%%%%%%%%%%%%%%%%%%%%%
%%% tensor network drawing tools
%%% taken in part from arxiv.org/abs/1603.03039

\usepackage{xcolor}
\definecolor{tensorblue}{rgb}{0.8,0.8,1}
\definecolor{tensorred}{rgb}{1,0.5,0.5}
\definecolor{tensorpurp}{rgb}{1,0.5,1}
\definecolor{tensorgreen}{rgb}{0.3,0.8,0.3}

\usepackage{tikz}
\newcommand{\diagram}[1]{
  \begin{tikzpicture}
    [scale=.5, every node/.style = {sloped,allow upside down},
    baseline = {([yshift=+0ex]current bounding box.center)},
    rounded corners=2pt]
    #1
  \end{tikzpicture}}

\tikzset{tens/.style={fill=tensorblue}}
\tikzset{prp/.style={fill=tensorgreen}}
\tikzset{msr/.style={fill=tensorred}}
\tikzset{tengray/.style={fill=black!20}}

\usetikzlibrary{calc}
\newcommand{\wire}[4][]{
  \draw[#1] (#3,#2) -- (#4,#2)
}
\newcommand{\vwire}[4][]{
  \draw[#1] (#2,#3) -- (#2,#4)
}
\newcommand{\rect}[6][tens]{
  \draw[#1] (#4,#2) rectangle (#5,#3);
  \draw ($ 0.5*(#4,#2) + 0.5*(#5,#3) $) node {$#6$}
}
\renewcommand{\dot}[2]{
  \node at (#2,#1) [circle,fill,inner sep=1.5pt]{};
}
\renewcommand{\cross}[3][.25]{
  \draw (#3,#2) circle (#1) node {};
  \draw ($ (#3,#2) - (#1,0) $) -- ($ (#3,#2) + (#1,0) $) node{};
  \draw ($ (#3,#2) - (0,#1) $) -- ($ (#3,#2) + (0,#1) $) node{};
}
\newcommand{\cnot}[3]{
  \dot{#2}{#1};
  \vwire{#1}{#2}{#3};
  \cross{#3}{#1};
}

% measurement meter
\newcommand{\meter}[5][tengray]{
  \rect[#1]{#2}{#3}{#4}{#5}{};
  \path[path picture={
    \draw[black] ([shift={(.1,.1)}]
    path picture bounding box.south west) to[bend left=50]
    ([shift={(-.1,.1)}]
    path picture bounding box.south east);
    \draw[black,-latex] ([shift={(0,.1)}]
    path picture bounding box.south)
    -- ([shift={(.2,-.1)}]path picture bounding box.north);}]
  (#4,#2) rectangle (#5,#3);
}


%%%%%%%%%%%%%%%%%%%%%%%%%%%%%%%%%%%%%%%%%%%%%%%%%%%%%%%%%%%%%%%%%%%%%%


\begin{document}
\count\footins = 1000 % play nicely with long footnotes

\title{Cutting and stitching quantum circuits}
\author{Michael A. Perlin}
\date{\today}

\maketitle
\thispagestyle{fancy}

\tableofcontents


\section{Introduction}
\label{sec:intro}

Every quantum computation can be represented by a quantum circuit.
This quantum circuit in turn defines a probability distribution over
the possible (classical) states of its qubits.  The purpose of
performing a quantum computation is essentially to sample the
distribution defined by a corresponding quantum circuit.  Some quantum
circuits, however, are too large to run directly on any available
hardware.  To address this problem, the authors of some recent
work\cite{peng2019simulating} developed a scheme to (i) ``cut'' large
quantum circuits into smaller {\it sub-circuits} (or {\it fragments})
that can executed on available hardware, and (ii) reconstruct the
probability distribution defined by a large quantum circuit from
probability distributions associated with its fragments.

In these notes, we present our own take on the main result in
ref.~[\citenum{peng2019simulating}], introducing a new formalism for
thinking about circuit cutting, fragments, and reconstruction.  In
addition to clarifying the equivalence between two cutting strategies
used in the circuit cutting groups at ANL, our formalism motivates a
few simple and natural improvements to the cutting scheme in
ref.~[\citenum{peng2019simulating}], yielding major reductions to the
runtimes and memory footprints necessary to simulate fragments and
reconstruct probability distributions.  After presenting these
improvements, we discuss the problem of {\it sampling} (as opposed to
reconstructing) a probability distribution defined by large circuits,
and suggest some potential resolutions to this problem.

We will assume that the reader is familiar with tensor network
representations of quantum circuits.  For brevity, we will generally
blur the distinction between a tensor network, the circuit it
represents (when applicable), and the state prepared by this circuit.
We will also blur the distinction between the dangling edges of a
tensor network and qubit degrees of freedom associated with these
dangling edges.


%%%%%%%%%%%%%%%%%%%%%%%%%%%%%%%%%%%%%%%%%%%%%%%%%%%%%%%%%%%%%%%%%%%%%%
\section{The general cut-and-stitch prescription}
\label{sec:networks}

For any quantum state $\ket\psi$ of $N$ qubits, a straightforward
resolution of the identity operator $I=\sum_{b\in\set{0,1}}\op{b}$ on
qubit $n$ implies that
\begin{align}
  \ket\psi = I_n \ket\psi
  \simeq \sum_{b\in\set{0,1}}
  \ket{b} \otimes \prescript{}{n}{\bk{b|\psi}},
  \label{eq:pure_identity}
\end{align}
where $\simeq$ denotes equality up to a permutation of tensor factors
(i.e.~qubit order).  Here $\prescript{}{j}{\bk{b|\psi}}$ is a
sub-normalized state of $N-1$ qubits acquired by projecting $\ket\psi$
onto state $\ket{b}$ of qubit $n$.  If the structure of a quantum
circuit that prepares a state $\ket\psi$ allows, the identity in
\eqref{eq:pure_identity} can be used to ``cut'' a circuit by inserting
a resolution of the identity operator $I$ at a location that splits
the circuit into two disjoint sub-circuits, or {\it fragments}.  The
state at the end of the circuit can then be decomposed as
\begin{align}
  \ket\psi \simeq \sum_{b\in\set{0,1}}
  \ket{\psi_1\p{b}} \otimes \ket{\psi_2\p{b}},
\end{align}
where $\ket{\psi_1\p{b}}$ and $\ket{\psi_2\p{b}}$ are (possibly
sub-normalized) states prepared by each circuit fragment when
projecting onto $\ket{b}$ or preparing $\ket{b}$ on the appropriate
wire of the circuit fragments.  A circuit that prepares the
three-qubit GHZ state, for example, can be cut as
\begin{align}
  \diagram{
    \wire{0}{-.5}{3.75};
    \wire{1.5}{-.5}{3.75};
    \wire{-1.5}{-.5}{3.75};
    \rect{1}{2}{0}{1}{H};
    \cnot{1.75}{1.5}{0};
    \cnot{2.75}{0}{-1.5};
  }
  ~ = \sum_{b\in\set{0,1}} ~
  \diagram{
    \wire{0}{-.5}{2.5};
    \wire{0}{5}{6.5};
    \wire{1.5}{-.5}{6.5};
    \wire{-1.5}{-.5}{6.5};
    \rect{1}{2}{0}{1}{H};
    \cnot{1.75}{1.5}{0};
    \rect[msr]{-.5}{.5}{2.5}{3.5}{b};
    \rect[prp]{-.5}{.5}{4}{5}{b};
    \cnot{5.5}{0}{-1.5};
  }
  ~ \simeq \sum_{b\in\set{0,1}} ~
  \diagram{
    \wire{0}{-.5}{3};
    \wire{1.5}{-.5}{3};
    \rect{1}{2}{0}{1}{H};
    \cnot{1.75}{1.5}{0};
    \rect[msr]{-.5}{.5}{2.5}{3.5}{b};
  }
  ~ \otimes ~
  \diagram{
    \wire{0}{-1}{1.5};
    \wire{1.5}{-.5}{1.5};
    \rect[prp]{1}{2}{-1.5}{-.5}{b};
    \cnot{0.5}{1.5}{0};
  }.
  \label{eq:cut_example}
\end{align}
In practice, however, a quantum computation does not yield the full
state $\ket\psi$, but rather measurement statistics from sampling the
diagonal elements of the density operator $\rho=\op\psi$ in some fixed
computational basis.  The identity analogous to
\eqref{eq:pure_identity} for density operators $\rho$ reads
\begin{align}
  \rho \simeq \f12 \sum_{M\in\B} M \otimes \tr_n\p{M_n\rho},
  \label{eq:core_identity}
\end{align}
where $\B\equiv\set{X,Y,Z,I}$ is the set of Pauli operators together
with the singe-qubit identity $I$; $\tr_j$ denotes the partial trace
with respect to qubit $n$; and $M_n$ denotes an operator which acts
with $M$ on qubit $n$ and trivially (i.e.~with the identity $I$) on
all other qubits.  If the structure of a circuit allows it to be cut
at one location into two fragments, The identity in
\eqref{eq:core_identity} implies that the density operator sampled
when measuring the circuit can be decomposed as
\begin{align}
  \rho \simeq \f12 \sum_{M\in\B} \rho_1\p{M} \otimes \rho_2\p{M}.
  \label{eq:cut_identity}
\end{align}
If $\rho_f\p{M}$ is a $Q_f$-qubit operator, then $\rho_f$ is a linear
map from the space of single-qubit operators to the space of
$Q_f$-qubit operators.  The map $\rho_f$ naturally admits a
representation as a tensor network with $Q_f+1$ dangling edges, which
maps single-qubit operators on the dangling {\it cut edge} $e_f$ of
$\rho_f$ to $Q_f$-qubit operators on the dangling {\it non-cut edges}
$\bar{\v c}_f$ of $\rho_f$.  The operator $\rho_f\p{M}$ is recovered
from $\rho_f$ by contracting the cut edge $e_f$ with a vector
representation of the operator $M$.

More generally, we can make $C$ cuts to split a circuit into $F$
fragments $\set{\rho_f}$.  In order to stitch all fragments together,
we will now need to sum over a list of operators
$\v M\equiv\p{M_1,M_2,\cdots,M_C}\in\B^C$, where the operator $M_k$ is
``assigned'' to stitch (or cut) $k$.  Denoting the set of all stitches
incident to fragment $\rho_f$ by $\v s_f$, we define
$\v M_f\equiv\p{M_s:s\in\v s_f}$ to be the restriction of $\v M$ to
the stitches in $\v s_f$; that is, $\v M_f$ is a list of the operators
in $\v M$ that are associated with stitches incident to fragment
$\rho_f$.  We can then expand the full operator $\rho$ as
\begin{align}
  \rho \simeq \f1{2^C} \sum_{\v M\in\B^C}
  \bigotimes_{f\in\Z_F} \rho_f\p{\v M_f}.
  \label{eq:general_identity}
\end{align}
Although we will keep \eqref{eq:general_identity} in mind as the
general prescription for reconstructing a circuit from multiple
fragments, for simplicity we will generally consider the single-cut
case in \eqref{eq:cut_identity} throughout this work.


%%%%%%%%%%%%%%%%%%%%%%%%%%%%%%%%%%%%%%%%%%%%%%%%%%%%%%%%%%%%%%%%%%%%%%
\section{Application to quantum circuits and runtime improvements}
\label{sec:circuits}

The expansion in \eqref{eq:cut_identity} provides a prescription to
cut and stitch {\it tensor networks}, but what about quantum circuits?
What does it mean to attach some operator $M\in\B$ to a circuit?  The
interpretation of such attachment is clarified by expanding Pauli
operators $B\in\set{X,Y,Z}$ in their diagonal basis as
$B=\sum_s s B_s$, where $B_s\equiv\op{B_s}$ is a density operator
corresponding to the pure state $\ket{B_s}$ satisfying
$B\ket{B_s}=s\ket{B_s}$, namely
\begin{align}
  \ket{Z_+} = \ket0, && \ket{Z_-} = \ket1, &&
  \ket{X_\pm} = \f{\ket0 \pm \ket1}{\sqrt2}, &&
  \ket{Y_\pm} = \f{\ket0 \pm i\ket1}{\sqrt2}.
\end{align}
Substituting these expansions into \eqref{eq:cut_identity} yields
\begin{align}
  \rho \simeq \f12 \sum_{\substack{B\in\set{X,Y,Z}\\r,s\in\set{\pm1}}}
  rs\, \rho_1\p{B_r} \otimes \rho_2\p{B_s}
  + \f12 \sum_{r,s\in\set{\pm1}} \rho_1\p{Z_r} \otimes \rho_2\p{Z_s},
  \label{eq:cut_identity_XYZ}
\end{align}
where we can now interpret the attachments of states $B_s$ similarly
to the attachments to circuits in \eqref{eq:cut_example}: either a
preparation of a qubit in the state $B_s$, or a post-selection on
finding a qubit in the state $B_s$, as appropriate.  Rather than
thinking about the operators $\rho$ and $\rho_f$, it will be helpful
to think about their diagonal elements $p\equiv\diag\rho$ and
$p_f\equiv\diag\rho_f$ in a fixed computational basis.

The distributions $p_f$ can be interpreted as {\it conditional}
\mbox{(quasi-)probability} distributions.  The ``input'' condition
$B_s$ defines a probability distribution $p_2\p{B_s}$ sampled by
\begin{enumerate*}
\item preparing the qubit $e_2$ at the cut edge of $\rho_2$ in the
  state $B_s$,
\item running the circuit defined by $\rho_2$, and
\item measuring all qubits in the computational basis\footnote{The
    preparation of a state $B_s$ on qubit $e_2$ is essentially the
    ``new method'' in ref.~[\citenum{peng2019simulating}] for
    preparing $\rho_2\p{B_s}$.  The ``old method'' for preparing
    $\rho_2\p{B_s}$ was to initialize a pair of qubits in the
    maximally entangled state $\p{\ket{00}+\ket{11}}/\sqrt2$, insert
    one of these qubits in place of $e_2$, and then post-select all
    results on measurement of the second qubit in the state $B_s$.
    Such post-selection projects onto an initial product state
    $B_s\otimes B_s$ of the maximally entangled qubits, and is
    therefore equivalent to simply preparing the first qubit in $B_s$
    and forgetting about the second qubit entirely, as in the ``new
    method''.}.
\end{enumerate*}
The ``output'' condition $B_r$ defines a distribution $p_1\p{B_r}$
acquired by
\begin{enumerate*}
\item running the circuit defined by $\rho_1$,
\item collecting the joint probability distribution
  $\tilde p_1\p{B_r}$ sampled by measuring qubit $e_1$ in the diagonal
  basis of $B$ and all other qubits in the computational basis, and
\item taking a ``slice'' of the joint probability distribution
  $\tilde p_1\p{B_r}$ at a fixed measurement outcome $B_r$ on qubit
  $e_1$.
\end{enumerate*}
Note that additionally normalizing the distribution $p_1\p{B_r}$ by
$\tr p_1\p{B_r}$, which is equal to the probability of finding qubit
$e_1$ in the state $B_r$, would yield a conditional probability
distribution; we need to keep $p_1\p{B_r}$ sub-normalized for the
reconstruction of $p$ from $p_1$ and $p_2$.

In order to express \eqref{eq:cut_identity_XYZ} and interpret the
conditional operators $\rho_f$, we expanded every instance of
$\rho_f\p{M}$ in \eqref{eq:cut_identity} in the diagonal basis of $M$.
For reasons that escape me, the authors of
ref.~[\citenum{peng2019simulating}] decided to expand $\rho_2\p{M}$ in
the diagonal basis of $M$ while leaving $\rho_1\p{M}$ intact, yielding
the reconstruction formula
\begin{align}
  p \simeq \f12 \sum_{\substack{B\in\set{X,Y,Z}\\s\in\set{\pm1}}}
  s\, p_1\p{B} \otimes p_2\p{B_s}
  + \f12\sum_{s\in\set{\pm1}} p_1\p{I} \otimes p_2\p{Z_s},
  \label{eq:cut_identity_probs_bad}
\end{align}
where the distributions $p_1\p{M}$ are acquired by computing
e.g.~$p_1\p{B_+}$ and $p_1\p{B_-}$, then taking a linear combination
to get $p_1\p{B}=p_1\p{B_+}-p_1\p{B_-}$.  In practice, the slowest
step in reconstructing the probability distribution $p$ from fragment
distributions $p_1$ and $p_2$ is the evaluation of tensor products
such as $p_1\p{B} \otimes p_2\p{B_s}$.  Rather than expanding
$p_2\p{M}$ in the diagonal basis of $M$, it is therefore always
favorable to compute $p_2\p{M}$ via classical pre-processing similarly
to $p_1\p{M}$, and combining these distributions according to
\begin{align}
  p \simeq \f12 \sum_{M\in\B} p_1\p{M} \otimes p_2\p{M}.
  \label{eq:cut_identity_probs}
\end{align}
For a circuit with $C$ cuts, combining fragment distributions
according to \eqref{eq:cut_identity_probs} rather than
\eqref{eq:cut_identity_probs_bad} reduces the number of tensor
products that must be evaluated from $8^C$ to $4^C$, which speeds up
reconstruction time by a factor of $\O\p{2^C}$.


%%%%%%%%%%%%%%%%%%%%%%%%%%%%%%%%%%%%%%%%%%%%%%%%%%%%%%%%%%%%%%%%%%%%%%
\section{Informational completeness and memory improvements}
\label{sec:completeness}

Using a circuit fragment $\rho_f$ with a single cut edge $e_n$ for
reconstruction according to \eqref{eq:cut_identity_probs} requires
characterizing the conditional distribution $p_f$ for several states
(conditions) $\sigma$.  Once we know $p_f\p{Z_+}$ and $p_f\p{Z_-}$,
for example, we can compute both $p_f\p{Z}=p_f\p{Z_+}-p_f\p{Z_-}$ and
$p_f\p{I}=p_f\p{Z_+}+p_f\p{Z_-}$.  How many states $\sigma$ do we need
to choose, or equivalently how many distributions $p_f\p{\sigma}$ do
we need to compute, in order to determine all distribution $p_f\p{M}$
that we need for reconstruction?  The recombination prescription in
\eqref{eq:cut_identity_probs_bad} uses six states: $B_s$ for each of
$s\in\set{\pm1}$ and $B\in\set{X,Y,Z}$.  If a fragment $\rho_f$ has
$s_f$ incident stitches, therefore, at face value recombination
prescription in \eqref{eq:cut_identity_probs_bad} requires
characterizing $6^{s_f}$ distributions associated with fragment
$\rho_f$.  In a general sense, the reason that the six states
$\set{B_s}$ are sufficient to characterize any conditional is that
they form a complete basis for the space of single-qubit operators; in
fact, these states form an {\it over}-complete basis, as only four of
them are linearly independent.  We therefore do not need to keep track
of distributions $p_f\p{\sigma}$ for all six states in $\set{B_s}$, as
we can use the fact that $I=B_++B_-$ for any $B\in\set{X,Y,Z}$ to
recover, for example,
\begin{align}
  p_f\p{X_-}
  = p_f\p{I} - p_f\p{X_+}
  = p_f\p{Z_+} + p_f\p{Z_-} - p_f\p{X_+}.
  \label{eq:X-_expansion}
\end{align}
We thus only need to characterize a conditional $p_f$ with an {\it
  informationally complete} set of conditions $\sigma$ that
collectively span the space of single-qubit operators; this space is
four-dimensional, which means that we should only need to characterize
$4^{s_f}$ distributions for a fragment $\rho_f$ with $s_f$ incident
stitches.

In the case of ``input'' conditions, as for the conditional $p_2$ in
\eqref{eq:cut_identity_probs}, the expansion in
\eqref{eq:X-_expansion} makes it clear that it suffices to prepare
each state $\sigma \in \set{X_+,Y_+,Z_+,Z_-}$, and store the
corresponding distributions $p_2\p{\sigma}$.  This choice of basis for
single-qubit operators, however, is ``biased'' along the $z$-axis,
which loosely speaking introduces unevenly distributed error in any
empirical characterization of the conditional $p_f$.  An unbiased
basis of states is provided by any {\it symmetric, informationally
  complete, positive operator-valued measure} (SIC-POVM), which for a
qubit consists of the states on the four corners of a regular
tetrahedron inscribed in the Bloch sphere.  The choice of orientation
for the tetrahedron is arbitrary; one such choice consists of unit
vectors $\uv v_j$ proportional to
\begin{align}
  \v v_1 \equiv \p{ +1, +1, +1 }, &&
  \v v_2 \equiv \p{ +1, -1, -1 }, &&
  \v v_3 \equiv \p{ -1, +1, -1 }, &&
  \v v_4 \equiv \p{ -1, -1, +1 },
\end{align}
which define states $S_j\equiv\p{I+\uv v_j\c\v X}/2$ with
$\v X\equiv\p{X,Y,Z}$.  Knowing the distribution $p_f\p{\sigma}$ for
all states $\sigma\in\set{S_j}$ allows us to compute $p_f\p{M}$ for
any single-qubit operator $M=M_0I+\v M\c\v X$ via the expansion
\begin{align}
  M = \f12 \sum_j \p{ M_0 + 3 \v M\c\uv v_j } S_j
  &&
  \implies
  &&
  p_f\p{M} = \f12 \sum_j \p{ M_0 + 3 \v M\c\uv v_j } p_f\p{S_j}.
  \label{eq:SIC_expansion}
\end{align}

In the case of ``output'' conditions, as for the conditional $p_1$ in
\eqref{eq:cut_identity_probs}, ideally we would perform a SIC-POVM
measurement directly.  While there has been some
theoretical\cite{tabia2012experimental} and
experimental\cite{bent2015experimental, bian2015realization} progress
in implementing direct SIC-POVM measurements, for now we are limited
to making projective measurements of qubits in some basis of two
mutually orthogonal states.  Making projective measurements of
SIC-POVM basis elements would therefore require measuring the qubit at
edge $e_1$ of fragment $\rho_1$ in four different bases, which can
yield one of eight different outcomes.  Four of these outcomes prepare
a SIC-POVM state $S_j$, which in turn prepares the state (proportional
to) $\rho_1\p{S_j}$ at the non-cut edges $\bar{\v e}_1$ of fragment
$\rho_1$; measuring the qubits at $\bar{\v e}_1$ then samples the
distribution $p_1\p{S_j}$.  The problem with this measurement
strategy, however, is that the remaining possible measurement
outcomes, $\set{I-S_j}$, do not give us any (direct) information about
the four distributions $p_1\p{S_j}$ that we are trying to
characterize.  We must therefore either store {\it eight}
distributions per ``output'' condition, or throw out half of our
possible measurement outcomes.

The resolution to this dilemma is simple: we simply go back to
measuring in the three eigenbases of Pauli operators $X$, $Y$, $Z$.
Rather than keeping track of one distribution for each of six possible
measurement outcomes, however, we can (i) keep track of the
distributions $p_1\p{B_+}$ for each $B\in\set{X,Y,Z}$ as appropriate,
i.e.~add to a histogram empirically characterizing $p_1\p{B_+}$ after
finding the measured qubit at output edge $e_1$ in the state $B_+$,
and then (ii) {\it always} collect data on the distribution $p_1\p{I}$
{\it regardless} of the measurement outcome, as
\begin{align}
  I = \f13 \sum_{\substack{B\in\set{X,Y,Z}\\s\in\set{\pm1}}} B_s
  &&
  \implies
  &&
  p_f\p{I} = \f13 \sum_{\substack{B\in\set{X,Y,Z}\\s\in\set{\pm1}}}
  p_f\p{B_s}.
\end{align}
This measurement scheme is economical in the sense that it (i) stores
four probability distributions per ``output'' condition (i.e.~the
minimum number necessary for full reconstruction), and (ii) makes use
of all possible measurement outcomes.\footnote{Note that making
  projective measurements in only two different bases is insufficient
  to characterize the conditional distribution $p_1$.  Although
  measurements in two different bases yield four possible outcomes,
  these outcomes are not linearly independent, as can be seen from
  \eqref{eq:X-_expansion}.}  Knowing the distribution $p_f\p{\sigma}$
for all conditions $\sigma\in\set{I,X_+,Y_+,Z_+}$ allows us to compute
$p_f\p{M}$ for any single-qubit operator $M = M_0 I + \v M\c\v X$ with
$\v M\equiv\p{M_X,M_Y,M_Z}$ via the expansion
\begin{align}
  M = 2 \v M\c\v X_+ + \p{ M_0 - \sum_{B\in\set{X,Y,Z}} M_B } I
\end{align}
where $\v X_+\equiv\p{X_+,Y_+,Z_+}$ and (for reference)
$B_+=\p{I+B}/2$.


%%%%%%%%%%%%%%%%%%%%%%%%%%%%%%%%%%%%%%%%%%%%%%%%%%%%%%%%%%%%%%%%%%%%%%
\section{The sampling problem}
\label{sec:sampling}

-- TO BE CONTINUED --


\bibliography{\jobname.bib}

\end{document}

todo:

put notes on Overleaf, share with Martin + Brad

better/more motivation
clarify implications

sampling -- connection to noise on real device?

which circuits does this apply to?
- HWEA
- anything else?
